\documentclass[10pt,a4paper]{article}
\usepackage[utf8]{inputenc}
\usepackage{amsmath}
\usepackage{amsfonts}
\usepackage{amssymb}
\usepackage{amsthm}
\usepackage{mathtools} % \coloneqq  ( tlmgr install mathtools )
% \usepackage[lite]{mtpro2}
\theoremstyle{definition}
\newtheorem{definition}{Definition}[section]

% \theoremstyle{lemma}
\newtheorem{lemma}{Lemma}[section]

% \theoremstyle{theorem}
\newtheorem{theorem}{Theorem}[section]

\newtheorem{exercise}{Exercise}[section]
\newcommand{\Fun}{{\mbox{Fun}}}
\newcommand{\Rel}{{\mbox{Rel}}}
\newcommand{\dom}{{\mbox{dom}}}
\newcommand{\apo}{{\mbox{'}}}  % apostrophe
\newcommand{\The}{{\mbox{The\,}}}
\newcommand{\Ob}{{\mbox{Ob}}}
\newcommand{\Mor}{{\mbox{Mor}}}
\newcommand{\Cat}{{\mbox{Cat}}}
\newcommand{\Hom}{{\mbox{Hom}}}
\newcommand{\Nat}{{\mbox{Nat}}}
\newcommand{\HomMor}{{\widetilde{\Hom}\mbox{}}}
\newcommand{\FMor}{{\widetilde{F}\mbox{}}}
\newcommand{\GMor}{{\widetilde{G}\mbox{}}}
\newcommand{\op}{{\mbox{op}}}
\newcommand{\id}{{\mbox{id}}}
\newcommand{\mor}{{\mbox{mor}}}
\newcommand{\LRA}{\Longleftrightarrow}
\newcommand{\defi}{{\mbox{def}}}
\newcommand{\eqdef}{{\stackrel{\defi}{=}}}
\newcommand{\propdef}{{\stackrel{\defi}{\ \Longleftrightarrow\ }}}
\newcommand{\inter}{{\bigcap}}
\newcommand{\interclass}{{{\bigcap}_C}}
\newcommand{\Set}{{\mbox{Set}}}
\newcommand{\myprf}{\noindent\textbf{Proof.}}
\newcommand{\myqed}{\noindent\textbf{Qed.}}

\usepackage{float} % H hardposition for tables ( tlmgr install float )
\begin{document}
Some Proofs of the Category Theory\\
Georgy Dunaev, georgedunaev@gmail.com\\
\section{Logic and Set Theory}
\begin{table}[H]
\centering
\caption{Axiom schemata of predicate calculus with equality}
%\caption{Схемы аксиом логики предикатов с равенством}
\label{tab:axiompltab}
\begin{tabular}{|c|c|c|c|}
\hline
% назвывние & утверждение & условие\\ \hline
name & statement & condition & parameters(meta)\\ \hline
\hline
impI & $A \to (B \to A)$ & - & $A,B \in \mbox{Fm}$\\ %\hline
impE & $(A \to (B \to C))\to((A\to B)\to(A\to C))$ & -& $A,B,C \in \mbox{Fm}$\\ \hline
\hline
andE1 & $A \land B \to A$& -& $A,B \in \mbox{Fm}$\\ %\hline
andE2 & $A \land B \to B$& -& $A,B \in \mbox{Fm}$\\ %\hline
andI & $A\to(B\to (A \land B))$& -& $A,B \in \mbox{Fm}$\\ \hline
\hline
orI1 & $A \to  A \lor B$& -& $A,B \in \mbox{Fm}$\\ %\hline
orI2 & $B \to  A \lor B$& -& $A,B \in \mbox{Fm}$\\ %\hline
orE & $(A \to  C)\to ((B \to  C) \to  (A \lor B \to  C))$& -& $A,B,C \in \mbox{Fm}$\\ \hline
\hline
negI & $(A \to  B) \to  ((A \to  \neg B) \to  \neg A)$& -& $A,B \in \mbox{Fm}$\\ %\hline
negE & $A \to  (\neg A \to  B)$& -& $A,B \in \mbox{Fm}$\\ \hline
\hline
LEM & $A \lor \neg A$& -& $A \in \mbox{Fm}$\\ \hline
\hline
$\forall$E & $\forall xA\to A(t/x)$ & $\mbox{FFI}(t,x,A)$& $A \in \mbox{Fm}$\\ %\hline
$\exists$I & $A(t/x)\to \exists x A$ & $\mbox{FFI}(t,x,A)$& $A \in \mbox{Fm}$\\ \hline
\hline
=refl&$\forall w (w=w)$&-& $w \in \mbox{Var}$\\
=subs&$(x=y)\to ( A(x/y)\to A) $&$\mbox{FFI}(x,y,A)$& $A \in \mbox{Fm}$\\\hline
\end{tabular}
\end{table}
%Схема аксиом логики предикатов без равенства получается, если удалить последние две аксиомы.
%Здесь мы обозначили как $\mbox{FFI}(t,x,A)$ утверждение ``Терм $t$ свободен для подстановки вместо переменной $x$ в формуле $A$'', и как $A(t/x)$ -- результат корректной подстановки терма t вместо x.\\

\begin{table}[H]
\centering
% Rules of predicate calculus
%\caption{Правила вывода логики предикатов} \label{tab:rulestab}
\begin{tabular}{|c|c|c|c|}
\hline
name & rule & condition & parameters \\ \hline
%название & правило & условие \\ \hline
\hline
MP &  \rule[-2.5ex]{0pt}{7ex} $\frac{A\to B\quad A}{B}$ & - \\ \hline %(\mbox{MP})
\hline
%\hline
%\infer[(\forall\mbox{I})]{A\to \forall x B}{A\to B}, если $x \notin \mbox{FV}(A)$\\
\rule[-2.5ex]{0pt}{7ex}
$\forall$I & $\frac{A\to B}{A\to \forall x B}$ & $x \notin \mbox{FV}(A)$\\ \hline %(\forall\mbox{I})
\hline
%\hline
\rule[-2.5ex]{0pt}{7ex}
$\exists$E & $\frac{A\to B}{\exists x A\to B}$ & $x \notin \mbox{FV}(B)$\\ %(\mbox{$\exists$E})
\hline
\end{tabular}
\end{table}

\subsection{Intersections}
\begin{definition} Class intersection
$$\interclass K \propdef \{x: \forall k\in K. x\in k\}$$
\end{definition}

\begin{exercise}
$K\neq \emptyset \to \interclass K \in \Set$
\end{exercise}

\begin{exercise}
$K=\emptyset \to \interclass K = \Set$
\end{exercise}

\begin{definition} Set intersection
$$\inter K \propdef \{x: (\forall k\in K. x\in k)\land (\exists k.k\in K)\}$$
\end{definition}

\begin{exercise}
$\inter K \in \Set$
\end{exercise}
\subsection{Binders}
\begin{definition} Uniqueness quantifier
$$!x. P(x) \propdef \forall x_1 \forall x_2. P(x_1)\land P(x_2)\to x_1=x_2$$
\end{definition}

\begin{definition} ``Exists and unique'' quantifier
$$\exists !x. P(x) \propdef (\exists x. P(x))\land(!x. P(x))$$
\end{definition}

\begin{definition}  Dual for previous quantifier 
$$?x. P(x) \propdef \neg !x. \neg P(x)$$
\end{definition}

\begin{exercise}
$!x. P(x) \LRA \neg ?x. \neg P(x)$
\end{exercise}

\begin{definition} Description binder
$$\The x. P(x) \propdef \inter \{x:P(x)\ \land\ !w.P(w)\}$$
\end{definition}

\begin{exercise}
$(P(a)\ \land\ !w.P(w)) \rightarrow (\The x. P(x))=a$
\end{exercise}

\begin{exercise}
$(\exists !w.P(w)) \rightarrow P(\The x. P(x))$
\end{exercise}

\subsection{Values of functions}
Notation: $g(x)$ also means $(g\apo x).$\\

\begin{exercise}
$x\notin \dom(g)\to (g\apo x)=\emptyset$
\end{exercise}

\subsection{Extensionality}
\begin{theorem}
Assume $\forall~x\in X. g_1(x)=g_2(x)$, $\Rel(g_1)$, $\Fun(g_1)$,$\Fun(g_2)$,\\ $\dom(g_1)\subseteq\dom(g_2)$ and show that $g_1\subseteq g_2$.\\
\myprf\\
$p\in g_1$.
$\stackrel{\Rel(g_1)}{\Longrightarrow}$
There exists $a$, $b$ such that $p=\langle a, b\rangle$.
$\stackrel{p\in g_1}{\Longrightarrow}$ $\langle a, b\rangle\in g_1$.
\\
... to be continued ...
\\\myqed 
\end{theorem}

\section{Category Theory}
Notes and notation:\\
$\Ob_C$ and $C$ are used interchangeable. (e.g. $\Ob_\Set$ and $\Set$.)\\
Duality: $\Ob_C = \Ob_{C^\op}$ and 
 $\forall x,y \in \Ob_C. \Hom_C(x,y) = \Hom_{C^\op}(y,x)$.\\
$\circ : \Pi(C\in \Cat).\Pi(X,Y,Z\in C).\Hom_C(X,Y)\times\Hom_C(Y,Z)\to\Hom_C(X,Z)$

\subsection{Yoneda embedding lemma}
Let C be a locally small category.\\
Let Set be a category of sets.\\
$[C^{op},\mbox{Set}]$ is some functor category.\\

\begin{definition}
Let $C$, $D$ be the categories.\\
 $(F, \FMor)$ is a (covariant) functor iff\\
\begin{enumerate}
\item $F:C \to D$
\item $\FMor:\Pi(A\ B:C).\Hom_C(A,B) \to \Hom_D(F(A), F(B))$
\item $\forall A\in \Ob_C, \FMor^{A,A}(1_A)=1_{F(A)}$
\item $\forall X,Y,Z \in \Ob_C.\forall f\in\Hom_C(Y,Z).\forall g\in\Hom_C(X,Y). \FMor^{X,Z}(f\circ_C^{X,Y,Z} g)=\FMor^{Y,Z}(f)\circ_D^{F(X),F(Y),F(Z)} \FMor^{X,Y}(g)$
\end{enumerate}

\end{definition}

\begin{definition}
Let $C$, $D$ be the categories.\\
$[C,D]$ is the set of functors from $C$ to $D$.\\
\end{definition}

\begin{definition} The morphism function of the Hom-functor.\\
Let $f\in\Hom_{C}(A,B)$.
$$\Hom_C: \Ob_C \times \Ob_C \to \Set$$
$$\Hom_C(-,x): \Ob_C \to \Set$$
$$\HomMor^{A,B}_C(f,x) : \Hom_C(B,x) \to \Hom_C(A,x)$$
$$\HomMor^{A,B}_C(f,x) \eqdef (g \in\Hom_C(B,x) \mapsto (g \circ_C f))$$
$$\HomMor^{A,B}_C(f,x) \eqdef (g \in\Hom_C(B,x) \mapsto (g \circ_C f) \in\Hom_C(A,x))$$
$$\left(\mbox{another form is }\widetilde{\Hom_C(-,x)}^{A,B}(f) \eqdef (g \in\Hom_C(B,x) \mapsto (g \circ_C f) \in\Hom_C(A,x)\right)$$
\end{definition}

\begin{lemma}
$$\Hom_C(-, x) \in [C^\op,\Set]$$
\end{lemma}

%\noindent Proof.\\
\myprf\\
%\begin{itemize}
%\end{itemize}
Substitution:\\
$C \coloneqq C^\op$, 
$D \coloneqq \Set$, 
$F \coloneqq \Hom_C(-,x)$, 
$\FMor \coloneqq \HomMor_C(-,x)$\\
\begin{enumerate}
\item $\Hom(-, x) : C^\op \to \Set$\ \ \ by definition of $\Hom$.
\item $\HomMor(-, x) :\Pi(A\ B:C).\Hom_{C^\op}(A,B) \to \Hom_\Set(\Hom_C(A,x), \Hom_C(B,x))$\\
$\HomMor_C^{A,B}(f\in\Hom_{C^\op}(A,B), x) = (g \in\Hom_C(A,x) \mapsto (g \circ_C f) \in\Hom_C(B,x))$\ (this is a definition).\\
In other words, fix the $A,B\in C$ and $f\in\Hom_{C^\op}(A,B)$.\\
$\HomMor_C^{A,B}(f, x) = (g\mapsto (g \circ_C f))$\\
$\HomMor_C^{A,B}(f, x) : \Hom_C(A,x) \to \Hom_C(B,x))$
\\
$\HomMor$ is total since $f:\Hom_{C^\op}(A,B), \Rightarrow f\in\Hom_C(B,A)$ by definition of dual category. (It follows that composition is defined, so it is total.)
\item $1_A\in\Hom_{C^\op}(A,A)$\\
$\HomMor^{A,A}(1_A, x) = \left(g \in\Hom_C(A,x)\mapsto g \circ_C 1_A\right) = $\\
$\quad = \left(g \in\Hom_C(A,x) \mapsto g \in\Hom_C(A,x)\right) =$\\
$\quad \quad = 1_{\Hom_C(A,x)} \in \Hom_\Set\left(\Hom_C(A,x), \Hom_C(A,x)\right).$\\
since $1_A \circ_C g = g$\\
\item Now we need do a substitution in
$\forall X,Y,Z \in \Ob_C.\forall f\in\Hom_C(Y,Z).\forall g\in\Hom_C(X,Y). \FMor^{X,Z}(f\circ_C^{X,Y,Z} g)=\FMor^{Y,Z}(f)\circ_D^{F(X),F(Y),F(Z)} \FMor^{X,Y}(g)$ and prove it.\\
Again, $C \coloneqq C^\op$, 
$D \coloneqq \Set$, 
$F \coloneqq \Hom_C(-,x)$, 
$\FMor \coloneqq \HomMor_C(-,x)$\\
So $\FMor^{X,Z}(f\circ_C^{X,Y,Z} g)=\FMor(f)\circ_D^{F(X),F(Y),F(Z)} \FMor(g)$ becames\\
$\HomMor_C^{X,Z}(f\circ_C^{X,Y,Z} g, x)=\HomMor_C(f,x)\circ_D^{F(X),F(Y),F(Z)} \HomMor_C(g,x)$.\\
Equivalently, $\HomMor_C^{X,Z}(f\circ_{C^\op}^{X,Y,Z} g, x)=$\\ $=\HomMor_C^{Y,Z}(f,x)\circ_\Set^{\HomMor_C(X,x),\HomMor_C(Y,x),\HomMor_C(Z,x)} \HomMor_C^{X,Y}(g,x)$.\\
We'll omit some notation for arrow composition below.\\
$\HomMor_C^{X,Z}(f\circ_{C^\op}^{X,Y,Z} g, x)=\HomMor_C(f,x)\circ_\Set \HomMor_C(g,x)$,\\
where $f\in\Hom_{C^\op}(Y,Z)$ and $g\in\Hom_{C^\op}(X,Y)$.\\
So $f\in\Hom_{C}(Z,Y)$ and $g\in\Hom_{C}(Y,X)$.\\
$f\circ_{C^\op}^{X,Y,Z} g = g\circ_{C}^{Z,Y,X} f$\\
So it's enough to prove that $\HomMor_C(g\circ_C f, x) = \HomMor_C(f, x) \circ_\Set \HomMor_C(g, x)$\\
$a)\ \HomMor_C(g\circ_C f, x) = (h \in\Hom_C(Z,x) \mapsto (h \circ_C (g\circ_C f))\in\Hom_C(X,x))$\\
$b)\ \HomMor_C(f, x) = \left(u \in\Hom_C(Z,x) \mapsto (u \circ_C f)\in\Hom_C(Y,x)\right)$\\
$c)\ \HomMor_C(g, x) = \left(v \in\Hom_C(Y,x) \mapsto (v \circ_C g)\in\Hom_C\left(X,x\right)\right)$\\
\\
$u \stackrel{\HomMor_C(g, x)}{\mapsto\mapsto\mapsto\mapsto\mapsto} (u \circ_C g)\stackrel{\HomMor_C(g, x)}{\mapsto\mapsto\mapsto\mapsto\mapsto} ((u \circ_C g) \circ_C f))$
\\
$\Hom_C(Z,x) \stackrel{\HomMor_C(g, x)}{\to\to\to\to\to} \Hom_C(Y,x)\stackrel{\HomMor_C(f, x)}{\to\to\to\to\to} \Hom_C(X,x)$
\\
\end{enumerate}
\myqed
\begin{definition}
Yoneda functor between categories $C$ and $[C^\op,\Set]$.
$$h_x = \Hom_C(-, x)$$
\end{definition}


\begin{definition}
$$\eta : \Pi(w\in C^\op).\Hom_C(w,x)\to \Hom_C(w,y)$$
$$\eta = \lambda (w\in C^\op). \lambda g\in\Hom_C(w,x).(f\circ_C^w,x,y g)$$
\end{definition}
Now our aim is to define a natural transformation of the elements of $[C^\op,\Set]$.\\
\begin{lemma}$\mbox{ }$\\
We have:\\
$F=\Hom_C(-,x)\in [C^\op,\Set]$\\
$G=\Hom_C(-,y)\in [C^\op,\Set]$\\
$f\in\Hom_C(x,y)$\\
We want:\\
Find a natural tranformation $\eta$ from $\Hom_C(-,x)$ to $\Hom_C(-,x)$.\\ (define and prove)\\
\end{lemma}
Fix $w\in C^\op$. $\eta_w = ?$.\\
$\eta_w\Hom_\Set(\Hom_C(w,x),\Hom_C(w,y))$\\
$\eta_w: \Hom_C(w,x)\to \Hom_C(w,y)$\\

% prove that 
Substitution:\\
$C \coloneqq C^\op$, $D \coloneqq \Set$, $F \coloneqq \Hom_C(-,x)$, $G \coloneqq \Hom_C(-,y)$, $x \coloneqq a$, $y \coloneqq b$, $f \coloneqq m$
\begin{lemma}
$$\forall a,b\in C^\op. \forall m\in \Hom_C\op (a,b).\eta_b \circ_\Set \Hom_C(m,x)=\Hom_C(m,y)\circ_\Set\eta_a$$
\end{lemma}
\myprf\\
Fix $a$, $b$ and $m\in\Hom_C(b,a)$.\\
Aim: $(\lambda g_1 \in \Hom_C(b,x). f\circ^{b,x,y}_C g_1)\circ_\Set \HomMor_C^{B,A}(m,x) = $\\
$=\HomMor_C^{B,A}(m,y) \circ_\Set (\lambda g_2 \in \Hom_C(a,x). f\circ^{a,x,y}_C g_2)$.\\
Note that $(f\circ^{a,x,y}_C g_2) \in \Hom_C(a,y)$\\
$\HomMor_C^{B,A}(m,y) \eqdef (\lambda g_3 \in \Hom_C(a,y). g_3\circ^{b,a,y}_C m)$\\
$\HomMor_C^{B,A}(m,y) : \Hom_C(a,y) \to \Hom_C(b,y)$\\
RightHandSide $= \lambda g_2 \in \Hom_C(a,x). ((f \circ_C^{a,x,y} g_2) \circ_C^{b,a,y} m )$\\
RHS:$\Hom_C(a,x) \to \Hom_C(b,y)$\\
$\HomMor_C^{B,A}(m,x) \eqdef (\lambda g_4 \in \Hom_C(a,y). g_4\circ^{b,a,y}_C m)$\\
LeftHandSide $= \lambda g_2 \in \Hom_C(a,x). (f \circ_C^{b,x,y} (g_2 \circ_C^{b,a,x} m ))$\\
LHS:$\Hom_C(a,x) \to \Hom_C(b,y)$\\
LHS=RHS\\
\myqed
\begin{lemma}
$h_{-}:C\rightarrow [C^{op},\Set]$ is a fully faithful functor.
\end{lemma}
\myprf
\begin{itemize}
\item $h_-$ is a functor.\\
$\widetilde{h}_-$ = ...\\
Substitution: $C\coloneqq C$, $D\coloneqq [C^\op,\Set]$, $F\coloneqq h_-$,
\begin{enumerate}
\item $F:C \to D$
\item $\FMor:\Pi(A\ B:C).\Hom_C(A,B) \to \Hom_D(F(A), F(B))$
\item $\forall A\in \Ob_C, \FMor^{A,A}(1_A)=1_{F(A)}$
\item $\forall X,Y,Z \in \Ob_C.\forall f\in\Hom_C(Y,Z).\forall g\in\Hom_C(X,Y). \FMor^{X,Z}(f\circ_C^{X,Y,Z} g)=\FMor^{Y,Z}(f)\circ_D^{F(X),F(Y),F(Z)} \FMor^{X,Y}(g)$
\end{enumerate}
\item $h_-$ is full.
\item $h_-$ is faithfull.
\item $h_-$ is injective on morphisms.
\end{itemize}
\myqed
\begin{lemma}{by N.~Yoneda. (Contravariant version)}\\
$G : C^\op \to \Set$\\
$h_A = \Hom_C(-,A)$\\
$h_A : C^\op \to \Set$\\
$\Nat(h_A,G)\sim G(A)$\\
(there is a bijection between value of functior G and natural transformations from $h_A$ to $G$)
\end{lemma}
\myprf\\
Let $u\in G(A)$. Let's define a natural transformation $\Phi$ from $h_A$ to $G$.\\
$\Phi:\Pi(x\in C^\op).h_A(x)\to G(x)$ (because it's a Hom in Set)\\
1) Let $\Phi^u_A(\id_A)\eqdef u$. \\
Now we need to define $\Phi_x(f)$ for arbitrary $x\in C$ and\\
$f\in h_A(x).$ $=\Hom_C(x,A).$ $=\Hom_{C^\op}(A,x). \Longrightarrow$\\
$\Longrightarrow$ therefore $\GMor ^{A,x}(f):G(A)\to G(x)$ is defined.\\
So the definition will be $\Phi^u_x(f)\eqdef (\GMor^{A,x}(f))(u)$.\\
Now we need to show the correctness of the definition.\\
$\Phi^u_A(\id_A)\eqdef (\GMor^{A,A}(f))(\id_A) = (\id_{G(A)})(u) = u$. The correctness is proved.
2) The next aim is proving commutativity of natural transformation with functor applied to morphisms.\\
$\forall x,y\in C^\op.\forall f\in \Hom_{C^\op}(x,y).(G(f)\circ_\Set \Phi_x) = (\Phi_y \circ_\Set (\Hom_C(f,A)))$
Fix  $x,y\in C^\op$ and $f\in \Hom_{C^\op}(x,y)$.\\
Aim: $(G(f)\circ_\Set \Phi_x) = (\Phi_y \circ_\Set (\Hom_C(f,A)))$.
Fix $g\in \Hom_C(x,A)$. Recall that $\Hom_C(f,A)=(-\circ^{y,x,a})$.\\
Aim: $(G(f)\circ_\Set \Phi_x)(g) = (\Phi_y \circ_\Set (\Hom_C(f,A)))(g)$.\\
Equivalently, $(G(f)\circ_\Set \Phi_x)(g) = \Phi_y (g\circ_C^{y,x,A} f)$.\\
LHS $\coloneqq (G(f)\circ_\Set \Phi_x)(g) =($def of $\Phi)= \GMor^{x,y}(f)\left(\left(\GMor^{A,x}\left(g\right)\right)\left(u\right)\right) =$\\
$ =\GMor^{A,y}\left(g\circ_C f\right)(u)$\\
Then since $g\circ_C f = f\circ_{C^\op} g$ and $\GMor(f\circ_{C^\op} g)=  \GMor(f)\circ_\Set \GMor(g)$ we have:\\
RHS $\coloneqq \Phi^u_y(g\circ f)=($def of $\Phi)=\GMor^{A,y}\left(g\circ_C f\right)(u)$\\
This LHS=RHS holds therefore we obtained a natural transformation $\Phi\in\Nat(h_A,G)$ using the $u\in G(A)$.\\
We already know that the relation between $G(A)$ and $\Phi\in\Nat(h_A,G)$ is functional and total on $G(A)$.
The next step is to prove its injectivity and surjectivity on $\Phi\in\Nat(h_A,G)$.\\
Injectivity: $u_1,u_2\in G(A)$, $u_1\neq u_2$.\\
$\Phi_{1,A}(\id(A))= u_1\neq u_2 = \Phi_{2,A}(\id(A))\quad \Longrightarrow \quad \Phi_1 \neq \Phi_2$. It's proved. \\
Surjectivity on $\Nat(h_A,G)$.:\\
Let $\Psi\in\Nat(h_A,G)$. Let's show that $\exists u\in G(A).\Phi^u=\Psi$.\\
$u\coloneqq \Psi_A(\id_A)$\\
$\Psi:\Pi(x\in C). h_A(x)\to G(x)$\\
$\Phi^u_x(f) \eqdef (\GMor^{A,x}(f))(u)$.\\
Fix $x\in C$ and $f\in\Hom_C(x,A)$. $\Psi_x(f)=\ ?$.
In other words, there is a $\Psi\in\Nat(h_A,G)$, and it is known that $\Psi_A(\id_A) = u$. Find $\Psi_x(f)$.\\
Firstly, $\id_A \in \Hom_C(A,A)$. $\Longrightarrow (\id_A \circ_C^{x,A,A} f) \in \Hom_C(x,A)$. $\Longrightarrow \Psi_x(\id_A \circ_C^{x,A,A} f)\in G(A)$. $\Longrightarrow \Psi_x(f)\in G(A)$.\\
Secondly, $\id_A \in \Hom_C(A,A)$. $\Longrightarrow \Psi_A(\id_A) \in G(A)$. Since $\Psi_A(\id_A)=u$ we have $G(f)(u)\in G(x)$.\\
Since $\Psi$ is a natural transformation we obtained that $\Psi_x(f)=G(f)(u),$ which is equal to $\Phi^u_x(f)$.\\
This equality holds for any argument $f\in \Hom_C(x,A)$ then $\forall x\in C. \Psi_x=\Phi^u_x$ by functional extensionality. Therefore the relation is surjective on $\Nat(h_A,G)$.
\myqed\\
\end{document}